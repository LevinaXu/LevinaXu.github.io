\textbf{University of California, Los Angeles} \textit{Research Internship} \hfill Sept 2024--January 2025\par  

\textit{Topics: Ai4Science, Dynamic System, Graph Neural ODE} \hfill Advisor:  \href{https://scholar.google.com/citations?hl=en&user=TQgOjK0AAAAJ}{Yizhou Sun}, \href{https://scholar.google.com/citations?view_op=list_works&hl=en&hl=en&user=UedS9LQAAAAJ}{Wei Wang} Work with:  \href{https://scholar.google.com/citations?user=SejA1zsAAAAJ&hl=en}{Zijie Huang}\par
\begin{itemize}
        \item Rethink Graph Neural ODE through the lens of the generalized coupled systems of the physical world, employing variational inference to enable causal intervention, learning the physics rules rather than overfitting the data.
\end{itemize}\par

\textbf{Emory University, Melody lab} \textit{Research Assistant} \hfill Jan 2024--Sept 2024\par  

\textit{Topics: Ai4Science, Epidemiology, Graph Learning} \hfill Advisor:  \href{https://scholar.google.com/citations?user=eWow24EAAAAJ&hl=en&oi=ao}{Wei Jin} Work with \href{https://scholar.google.com/citations?user=C-NftTgAAAAJ}{B. Aditya Prakash}, \href{https://scholar.google.com/citations?user=mOINlwcAAAAJ&hl=en}{Carl Yang}\par
\begin{itemize}
        \item Develop an epidemiology-aware neural ordinary differential equation method, to model physical-informed disease dynamics. Learn regional interactions in continuous time and against irregular epidemic-series data.
        \item Integrate machine learning into epidemic modeling with a comprehensive software: EpiLearn, offering tools for forecasting, source detection, and data simulation.
        \item Review the application of GNN in epidemic modeling, discuss traditional mechanics with deep learning.
\end{itemize}\par



% \textbf{Westlake University} \textit{Research Internship} \hfill Jun 2024--Now\par  

% \textit{Topics: Ai4Science, Genomics, Multigene Perturbation, Graph Learning} \hfill Advisor:  \href{https://scholar.google.com/citations?user=Y-nyLGIAAAAJ&hl=en&oi=ao}{Stan Z. Li} Work with:  \href{https://scholar.google.com/citations?user=aPKKpSYAAAAJ&hl=zh-CN}{Jun Xia}\par
% \begin{itemize}
%         \item Design a hypergraph-based transformer to model multigene perturbations by connecting genes that are shared with the same gene ontology hyperedge. 
% \end{itemize}\par

\textbf{Wuhan University, MARS lab} \textit{Research Assistant} \hfill Jan 2023--Sept 2024\par  

\textit{Topics: Robustness, Backdoor Attack, Graph Learning} \hfill Work with: \href{https://scholar.google.com/citations?user=RwlJNLcAAAAJ&hl=en&oi=ao}{Dacheng Tao}  \par
\begin{itemize}
        \item Propose a defense method based on graph topology energy that effectively mitigates graph backdoor attacks, and validated its effectiveness and robustness on five datasets under both IID and non-IID settings.
\end{itemize}\par

\textit{Topics: Domain Generalization, Federated Learning} \hfill Work with: \href{https://scholar.google.com.sg/citations?user=1LxWZLQAAAAJ&hl=en}{Qiang Yang}  \par
\begin{itemize}
        \item Develop a comprehensive software tool to benchmark three key issues for currently trustworthy federated learning: Generalization, Robustness, and Fairness. Write a Survey to discuss the status and future development.
\end{itemize}\par

\textit{Topics: Graph Learning, Domain Generalization, Federated Learning} \hfill Advisor: \href{https://scholar.google.com/citations?user=j-HxRy0AAAAJ&hl=en}{Mang Ye}  Work with  \href{https://scholar.google.com/citations?user=Shy1gnMAAAAJ&hl=en}{Bo Du}\par 
\begin{itemize}
        \item Identify the domain similarities from a novel graph spectrum perspective,  design a spectral sharing transformer and personalized adaptive convolution for domain-agnostic graph learning.
	\item Propose a framework to address the challenge of across-domain graph data shift in federated graph learning. 
	\item Propose a novel federated graph learning frame for both node and graph-level calibration, shedding good light on future research in solving the non-IID problem in federated graph learning scenarios.
\end{itemize}\par



\textbf{University of Notre Dame} \textit{Research Internship} \hfill Sept 2023--Jan 2024\par  

\textit{Topics: Inference Acceleration, Heterophilic Graph, Unsupervised Learning} \hfill Work with:  \href{https://scholar.google.com/citations?user=hDLBEhkAAAAJ}{Nitesh V Chawla}\par
\begin{itemize}
        % \item Discover heterophilic heterogeneity issues for real-world distributed graphs and design a general framework based on a proposed high-frequency area theorem. 
        \item Propose a unified framework to solve the heterophilic generalization and inference acceleration problems of graph self-supervised learning. Achieve good performance and 173x speedup for industrial-grade inference.
\end{itemize}\par


% \textbf{Wuhan University, National Innovative Programs} \textit{Research Membership} \hfill Sept 2022--Sept 203\par  

% \textit{Topics: Recommendation System, Graph Learning, Federated Learning} \hfill Advisor: Ming Zhong \par
% \begin{itemize}
% 	\item Proposed a feature decoupling approach to solve the non-IID problem in a federated recommendation system.

% \end{itemize}\par

